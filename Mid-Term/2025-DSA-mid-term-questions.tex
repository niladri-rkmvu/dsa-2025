\linespread{1.1}

% moderately liberal vertical spacing
\setlist[enumerate]{itemsep=0.6em,parsep=0pt,topsep=0pt}
\setlist[itemize]{itemsep=0pt,parsep=0pt,topsep=0pt}

% configure Verbatim to use uniform monospaced font
\fvset{fontfamily=tt,fontsize=\small,gobble=4}

\newcommand{\rmd}{\mathsf{d}}

% Custom section heading with right-aligned marks scheme
\newcommand{\sectionwithmarks}[2]{%
  \noindent\makebox[\textwidth][s]{\textbf{#1} \hfill \textbf{[#2]}}\\
  \noindent\rule{\textwidth}{0.4pt}%
}

\vspace{1em} % Adds a blank line before Section A
% -------------------------------
% Section A
% -------------------------------
\sectionwithmarks{Section A}{10 questions × 2 marks = 20 marks}

\begin{enumerate}[label=\arabic*. , leftmargin=*]

% -------------------------------
% Section A | Q.1 | 2 marks
% -------------------------------
\item What are the key differences between a compiled language and an interpreted language?  
Provide an example for each and discuss their advantages and disadvantages.

% -------------------------------
% Section A | Q.2 | 2 marks
% -------------------------------
\item Define a bit and a byte.  
If a computer uses 8-bit bytes, how many unique unsigned integer values can a single byte store?  
Using a diagram, illustrate how the value 7 would be stored in a byte of memory. Assume \texttt{int} consumes 2 bytes.

% -------------------------------
% Section A | Q.3 | 2 marks
% -------------------------------
\item Write down the characteristics of an algorithm.

% -------------------------------
% Section A | Q.4 | 2 marks
% -------------------------------
\item Arrange the functions in ascending order of their time complexity: \(2^n,\ n!,\ n^{\log n}\)

% -------------------------------
% Section A | Q.5 | 2 marks
% -------------------------------
\item What is Big \(\Theta\) notation? Provide a diagram and an example.

% -------------------------------
% Section A | Q.6 | 2 marks
% -------------------------------
\item Explain the Addition property of Big O notation and provide a proof.

% -------------------------------
% Section A | Q.7 | 2 marks
% -------------------------------
\item Find the time complexity of the following loop:\\
\hspace*{2em}\texttt{for (i = n; i >= 5; i = $\sqrt{i}$)}\\
\hspace*{4em}\texttt{x = x + 1;}

% -------------------------------
% Section A | Q.8 | 2 marks
% -------------------------------
\item Find the time complexity of the following loop:
\begin{Verbatim}[fontsize=\small,gobble=0]
    while n > 1 do
        for i = 1 to n do
            x = x + 1;
        end for
        n = floor(n / 2);
    end while
\end{Verbatim}

% -------------------------------
% Section A | Q.9 | 2 marks
% -------------------------------
\item We have an array \texttt{int A[-10..10][-20..20]}.  
Width of \texttt{int} = 2 bytes. Base Address = 1000.  
Find the address of \texttt{A[5][6]} in both Row Major Order (RMO) and Column Major Order (CMO).

% -------------------------------
% Section A | Q.10 | 2 marks
% -------------------------------
\item Write the \texttt{pop} functionality of a singly linked list.
\end{enumerate}

\vspace{1em} % Adds a blank line before Section A
% -------------------------------
% Section B
% -------------------------------
\sectionwithmarks{Section B}{5 questions × 4 marks = 20 marks}
\begin{enumerate}[label=\arabic*. , leftmargin=*]
% -------------------------------
% Section B | Q.1 | 4 marks
% -------------------------------
\item Solve the recurrence relation:
\[
T(n)=
\begin{cases}
1, & \text{if }n=0,\\
T\bigl(\tfrac{n}{2}\bigr)+T\bigl(\tfrac{2n}{5}\bigr)+7n, & \text{if }n>0.
\end{cases}
\]

% -------------------------------
% Section B | Q.2 | 4 marks
% -------------------------------
\item Consider an array \(A=\{3,8,2,5,7,6,12\}\) of length 7.
\begin{itemize}
  \item A subarray is a sequence of consecutive elements in the original array.
  \item A subarray of size \(w\) contains exactly \(w\) consecutive elements.
\end{itemize}
Let \(w=4\). Valid subarrays: \{3,8,2,5\}, \{8,2,5,7\}, \{2,5,7,6\}, \{5,7,6,12\}.  
Their sums are


\[
3+8+2+5 = 18,\quad
8+2+5+7 = 22,\quad
2+5+7+6 = 20,\quad
5+7+6+12 = 30.
\]


The maximum sum is \(30\).  
Write an optimal algorithm (time complexity \(<n^2\)) to find the maximum sum of any subarray of size \(w\).

% -------------------------------
% Section B | Q.3 | 4 marks
% -------------------------------
\item Design an algorithm to find the middle node of a singly linked list without using its length.  
Examples:

\begin{flushleft}
\noindent\(
10\;\rightarrow\;20\;\rightarrow\;30\;\rightarrow\;40\;\rightarrow\;50\;\rightarrow\;\texttt{None}
\)  
(\textbf{middle = 30})

\noindent\(
10\;\rightarrow\;20\;\rightarrow\;30\;\rightarrow\;40\;\rightarrow\;50\;\rightarrow\;60\;\rightarrow\;\texttt{None}
\)  
(\textbf{middle = 40})
\end{flushleft}

% -------------------------------
% Section B | Q.4 | 4 marks
% -------------------------------
\item Write an algorithm to reverse a singly linked list.  

\textbf{Original:}
\begin{flushleft}
\noindent\(
10\;\rightarrow\;20\;\rightarrow\;30\;\rightarrow\;40\;\rightarrow\;50\;\rightarrow\;\texttt{None}
\)
\end{flushleft}

\textbf{Reversed:}
\begin{flushleft}
\noindent\(
\texttt{None}\;\leftarrow\;10\;\leftarrow\;20\;\leftarrow\;30\;\leftarrow\;40\;\leftarrow\;50
\)
\end{flushleft}

% -------------------------------
% Section B | Q.5 | 4 marks
% -------------------------------
\item Write a recursive algorithm to compute \(\sin(x)\) using the series. Show the recursion tree starting with count = 3.

\[
\sin(x)=x-\frac{x^3}{3!}+\frac{x^5}{5!}-\frac{x^7}{7!}+\cdots
\]

Assume \(x\) in radians; terminate recursion by term count.
\end{enumerate}

\vspace{2em} % Adds 2 blank lines before Section C
% -------------------------------
% Section C
% -------------------------------
\sectionwithmarks{Section C}{2 questions × 10 marks = 20 marks}
% -------------------------------
% Section C | Q.1 | 10 marks
% -------------------------------
\begin{enumerate}[label=\arabic*. , leftmargin=*, itemsep=1em]
\item Illustrate with diagram the Tower of Hanoi problem for \(n=1,2,3\) disks, showing the sequence of moves.  
Write a recursive algorithm to illustrate the Tower of Hanoi problem.  
Determine its time complexity and why it becomes impractical for large \(n\).

% -------------------------------
% Section C | Q.2 | 10 marks
% -------------------------------
\item Let \(\texttt{int A[15]=\{0,1,2,\dots,14\}}\) and key = 5.  
\begin{enumerate}[label=(\roman*),itemsep=0pt,parsep=0pt]
  \item Write both iterative and recursive binary search algorithms; show each step in a table.
  \item Compute average times for successful and unsuccessful search.
\end{enumerate}
\end{enumerate}